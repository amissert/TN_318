
%-----------------------------------------------------------------------------
% Conclusions
%-----------------------------------------------------------------------------
\section{Conclusions}
\label{sec:conclusions}

A fit to the SK atmospheric neutrino data has been performed with the purpose
of estimating detector systematics parameters.  The fit is performed by
comparing the likelihoods for 288 distribution shape parameters and 37
atmospheric flux, cross section and normalization parameters using MCMC
methods. These fitted parameters are then used to perform toy MC studies. One
such study calculates the detector uncertainties for a set of signal and
background categories in different detector regions, and this is used to
calculate a figure of merit that determines the optimal fiTQun FV cuts.  A
second toy MC study calculates the overall detector systematics and covariances
for a set of visible event topologies in various visible energy regions.  This
uncertainty is then used as an input to additional SK systematics studies that
will feed into the T2K oscillation analyses using the fiTQun event selections.
The overall impact of the fiTQun event selections and FV cuts is described in
detail in TN-319~\cite{tn319} and features notable improvements in both sample
size and quality.


