
\section{Introduction}
\label{subsec:intro}

This technical note describes an analysis to determine both the optimal fiTQun
fiducial volume cuts and the detector systematics for the fiTQun-only T2K event
selection.  To achieve these goals, a fit to the Super Kamiokande (SK)
atmospheric neutrino data is performed using Markov chain Monte Carlo (MCMC)
techniques.  The result of this fit is a sampling of detector systematics
parameters that yield the best agreement between the SK atmospheric data and the
corresponding Monte Carlo simulated data (MC).

The detector systematics are then propagated to the fiTQun T2K event selections
using toy Monte Carlo studies where samples of the post-fit systematics parameters are
applied.  We can then quantify how these parameters affect the overall event
rates in each T2K sample, and this uncertainty is then quantified as a covariance matrix
and passed on to additional SK detector error studies and, eventually, the T2K
oscillation analyses.

By allowing the detector systematics parameters to vary independently in
different SK detector regions, we can determine how the systematic uncertainties
change near the inner detector (ID) boundary.  This information is then used to determine
the optimal fiducial volume (FV) cuts for the fiTQun-only T2K samples.

The structure of this analysis is largely based on the previous fits to SK atmospheric data
described in detail in, for example, T2K TN-186~\cite{tn186}.  However, to address the challenges of
estimating detector systematics in different detector regions, 
significant modifications have been made. The key differences include:
%
\begin{itemize}
  \item The parameterization of the detector uncertainties is expanded to allow
    different regions of the detector to have independent 
    parameters.
  \item The full distribution of each of the fiTQun cut variables is used in
    the fit, instead of just the number of events that pass or fail a particular
    cut (the ``core'' and ``tail'' samples in the language of TN-186).  
  \item An additional set of multiplicative parameters is used to give the fit
    further flexibility to account for data/MC differences in the shape of the
    cut variable distributions.
  \item Changes to the MCMC methods have been made to significantly speed up
    evaluation of each step and deal with the strong correlations present with
    the new parameterization
\end{itemize}
%
These features are discussed in detail in the following sections.



