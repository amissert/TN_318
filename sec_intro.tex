
\section{Introduction}
\label{subsec:intro}

This technical note describes an analysis to determine both the optimal fiTQun
fiducial volume cuts and the detector systematics for the fiTQun-only selection
of Super-Kamiokande neutrino events for the T2K experiment.  The fiTQun event
reconstruction algorithm uses a maximum-likelihood approach that is
substantially different from the previous event reconstruction methods that
have been used for the T2K analyses.  Details on the fiTQun reconstruction
algorithm and its performance can be found in technical notes
TN-146~\cite{tn146} and TN-153~\cite{tn153}.  The optimization of the fiTQun
topological cuts for  T2K events is described in TN-319~\cite{tn319}.  This
study of the fiTQun fiducial volume (FV) cuts is largely motivated by the
following observations:

\begin{itemize}
  \item The fiTQun algorithm features improved resolution of the neutrino interaction vertex.
  \item Studies of simulated data show the fiTQun reconstruction performance is not adversely affected
    in some detector regions rejected by the previous FV cut, which is 200 cm from the inner detector
    optical boundary.
  \item A rigorous optimization of the FV cuts in the presence of systematic uncertainties 
    has not been done before.
  \item Approximately 30\% of the SK inner detector mass is located outside of the 200 cm FV cut used in previous analyses.
  \item The T2K experimental sensitivity is currently limited by the statistical uncertainty in the SK samples.
\end{itemize}

The most significant obstacle to performing this study is the
estimation of the detector systematic uncertainties, which must be factored
into the optimization somehow, and may vary in the different detector regions.
To address this problem, a fit to the SK
atmospheric neutrino data is performed using Markov chain Monte Carlo (MCMC)
techniques.  The result of this fit is a sampling of detector systematics
parameters that yield the best agreement between the SK atmospheric data and
the corresponding Monte Carlo simulated data (MC).

The detector systematics are then propagated to the fiTQun T2K event selections
using toy Monte Carlo studies where samples of the post-fit systematics
parameters are applied.  By allowing the detector systematic uncertainty
parameters to vary independently in different SK detector regions, we can
estimate the systematic uncertainties in each region separately.  This allows
us to determine the impact of allowing events from the different detector
regions into the T2K samples.  We can then use this information to determine
the optimal fiducial volume (FV) cuts.  Once the fiTQun FV and topological cuts are
specified, a similar set of toy MC studies can be done to propagate the overall
SK detector uncertainties to the oscillation analyses.

The structure of this analysis is largely based on the previous fits to SK
atmospheric data described in detail in, for example, T2K TN-186~\cite{tn186}.
However, to address the challenges of estimating detector systematics in
different detector regions, significant modifications have been made. The key
differences include:
%
\begin{itemize}
  \item The parameterization of the detector uncertainties is expanded to allow
    different regions of the detector to have independent 
    parameters.
  \item The full distribution of each of the fiTQun cut variables is used in
    the fit, instead of just the number of events that pass or fail a particular
    cut (the ``core'' and ``tail'' samples in the language of TN-186).  
  \item An additional set of multiplicative parameters is used to give the fit
    further flexibility to account for data/MC differences in the shape of the
    cut variable distributions.
  \item Changes to the MCMC methods have been made to significantly speed up
    evaluation of each step and deal with the strong correlations present with
    the new parameterization
\end{itemize}
%
These features are discussed in detail in the following sections.



